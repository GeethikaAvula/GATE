\documentclass[a4paper ,10pt]{article}
\usepackage[left=1in, right=0.75in, top=1in, bottom=0.75in]{geometry}
\usepackage{graphicx} % Required for inserting images
\usepackage{siunitx}
\usepackage{setspace}
\usepackage{gensymb}
\usepackage{xcolor}
\usepackage{caption}
%\usepackage{subcaption}
\doublespacing
\singlespacing
\usepackage[none]{hyphenat}
\usepackage{amssymb}
\usepackage{relsize}
\usepackage[cmex10]{amsmath}
\usepackage{mathtools}
\usepackage{amsmath}
\usepackage{commath}
\usepackage{amsthm}
\interdisplaylinepenalty=2500
%\savesymbol{iint}
\usepackage{txfonts}
%\restoresymbol{TXF}{iint}
\usepackage{wasysym}
\usepackage{amsthm}
\usepackage{mathrsfs}
\usepackage{txfonts}
\let\vec\mathbf{}
\usepackage{stfloats}
\usepackage{float}
\usepackage{cite}
\usepackage{cases}
\usepackage{subfig}
%\usepackage{xtab}
\usepackage{longtable}
\usepackage{multirow}
%\usepackage{algorithm}
\usepackage{amssymb}
%\usepackage{algpseudocode}
\usepackage{enumitem}
\usepackage{mathtools}
%\usepackage{eenrc}
%\usepackage[framemethod=tikz]{mdframed}
\usepackage{listings}
%\usepackage{listings}
\usepackage[latin1]{inputenc}
%%\usepackage{color}{   
%%\usepackage{lscape}
\usepackage{textcomp}
\usepackage{titling}
\usepackage{hyperref}
%\usepackage{fulbigskip}   
\usepackage{tikz}
\usepackage{graphicx}
\lstset{
  frame=single,
  breaklines=true
}
\let\vec\mathbf{}
\usepackage{enumitem}
\usepackage{graphicx}
\usepackage{siunitx}
\let\vec\mathbf{}
\usepackage{enumitem}
\usepackage{graphicx}
\usepackage{enumitem}
\usepackage{tfrupee}
\usepackage{amsmath}
\usepackage{amssymb}
\usepackage{mwe} % for blindtext and example-image-a in example
\usepackage{wrapfig}
\graphicspath{{figs/}}
\providecommand{\mydet}[1]{\ensuremath{\begin{vmatrix}#1\end{vmatrix}}}
\providecommand{\myvec}[1]{\ensuremath{\begin{bmatrix}#1\end{bmatrix}}}
\providecommand{\cbrak}[1]{\ensuremath{\left\{#1\right\}}}
\providecommand{\brak}[1]{\ensuremath{\left(#1\right)}}
\usetikzlibrary{decorations.pathreplacing,backgrounds}
\usepackage{circuitikz}
\author{Geethika}

\begin{document}

\begin{enumerate}
\item Consider three $4$-variable functions $f_1, f_2, $and $f_3,$ which are expressed in sum-of-minterms as \newline \quad $f_1 = \sum\brak{0,2,5,8,14}$, \quad $f_2=\sum\brak{2,3,6,8,14,15}$, \quad $f_3 = \sum\brak{2,7,11,14}$ \newline For the following circuit with one AND gate and one XOR gate, the output function $f$ can be expressed as:
	\hfill(GATE-CS2019,30)
	\begin{figure}[H]
		\begin{align*}
			\begin{circuitikz}
    % Draw AND gate
    \draw (0,0) node[and port] (and) {AND};
    
    % Draw XOR gate
    \draw (4,-0.2) node[xor port] (xor) {XOR};
    
    % Connect AND output to XOR input
    \draw (and.out) -- (xor.in 1);
    
    % Draw input wires
    \draw (and.in 1) -- ++(-1,0) node[left] {$f_1$};
    \draw (and.in 2) -- ++(-1,0) node[left] {$f_2$};
    \draw (xor.in 2)--++(-90:2) --++(-5,0) node[left] {$f_3$};    % Draw XOR output
    \draw (xor.out) -- ++(1,0) node[right] {$f$};
\end{circuitikz}

		\end{align*}
	\end{figure}
		\begin{enumerate}
		\item $\sum\brak{7,8,11}$
		\item $\sum\brak{2,7,8,11,14}$
		\item $\sum\brak{2,14}$
		\item $\sum\brak{0,2,3,5,6,7,8,11,14,15}$
		\end{enumerate}
\end{enumerate}
\end{document}
